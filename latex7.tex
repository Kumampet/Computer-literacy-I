\documentclass[a4paper]{jarticle}
\title{A \LaTeX ~sample [Mathematics]}
\author{s1250050@u-aizu.ac.jp}
\date{\today}
\begin{document}
\maketitle
\section{数学記号の表示}
数学の式などの記号は equation 環境や eqnarray 環境を使用します。

サンプル
\begin{equation}
\int_a^{\infty} f(x) dx
\end{equation}
\begin{equation} \label{number}
y =\frac{a}{1+a} +
\frac{b}{1+\frac{b}{1+frac{b}{1+cdots}}} +
\sqrt{\sqrt{1+x}}
\end{equation}
式\ref{number}は...の様に引用する.
\begin{eqnarray*}
y &=& 1+2+3 \\
	&=& 1+5 \\
	&=& 6 \\
\end{eqnarray*}
\begin{equation} \label{matrix}
		y = \left(
			\begin{array}{rrr}
				1 & 2 & 3 \\
				4 & 5 & 6 \\
				7 & 8 & 9
			\end{array}
			\right)
\end{equation}
文章中に引用したいときにはドルマーク\$ ではさむと良いでしょう.
この様な $\sum_1^n n^2$ 使い方です.
\end{document}
