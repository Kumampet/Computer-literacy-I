\documentclass[a4paper]{jarticle}
\usepackage{tabularx}
\title{A \LaTeX ~sample [Tabular]}
\author{s1250050@u-aizu.ac.jp}
\date{\today}
\begin{document}
\maketitle
\section{Tabular}
\subsection{Tabular1}
この様に項目を表示できます。 \\
\begin{center}
\begin{tabular}{c c c || c}
りんご & みかん & ばなな & 果物 \\
じゃがいも & 大根 & レタス & 野菜 \\
豚 & 牛 & 鶏 & 肉 \\
\end{tabular}
\end{center}

\subsection{Tabular2}
表組に線を引くことができます。
\begin{tabular}{| l | r |  c | c |}
\hline
番号 & 名前 & 出身地 & 好物 \\
\hline
\hline
1 & 太郎 & アフリカ & リンゴ \\
\hline
2 & 花子 & 印度 & じゃがいも \\
\hline
\end{tabular}

\section{Tabularx}
Tabularx は横幅を均等割りできます。 \\
\begin{tabularx}{45zw}{| c || X |  X | X |}
\hline
番号 & 料理 & 材料(4皿分) & 一皿当りのカロリー(Kcal)\\
\hline
\hline
1 & カレー & 肉 200g、 玉ねぎ2、 人参2、 じゃがいも2、 市販のルーなど & 900 \\
\hline
2 & シチュー& 肉 200g、 玉ねぎ2、 人参2、 じゃがいも2、 市販のルーなど & 900 \\
\hline
3 & 肉じゃが & 肉 200g、 玉ねぎ2、 人参2、 じゃがいも2、 糸こんにゃく 200g、 醤油、 砂糖 & 900 \\
\hline
\end{tabularx}

\end{document}
