\documentclass[a4paper]{jarticle}
\usepackage{dvipdfmx}{graphicx}
% 画像ファイル(eps,png,jpeg,pdf) を取り込む場合には
% graphicx の使用を宣言します。

\title{A \LaTeX ~ sample [Post Script]}
\author{s1250050@u-aizu.ac.jp}
\date{\today}
\begin{document}
\maketitle
\section{Postscript の取り込み}
サンプル
\begin{figure}[bth]
	\begin{center}
		includegraphics[scale=0.5]{%
/home/course/lit1/pub/FIGURE/sylpheed-3.0.3.eps}
	end{center}
	\caption{sylpheed3 を開いたところ}
\end{figure}

figure と inculudegraphics のオプションの説明
\begin{center}
\begin{tabular}{c || l }
scale & 縦横同倍のスケール \\
hscale & 横方向のスケール \\
vscale & 縦方向のスケール \\
width=10zw & 横幅を全角 10 文字幅分にスケール\\
hight=10cm & 高さを 10cm にスケール\\
angle=90 & 90 度回転\\
origin=c & センターを中心に回転\\
h & その位置に表示\\
t & ページ上端に表示\\
b & ページ下端に表示\\
p & 単独ページに表示\\
\end{tabular}
\end{center}

あまり使わないけど、図だけでなく文字も回転できる。
\par
\rotaatebox{90}{\fbox{回転}}
\rotaatebox{60}{\fbox{回転}}
\rotaatebox{30}{\fbox{回転}}
\rotaatebox{10}{\fbox{回転}}
\rotaatebox{0}{\fbox{回転}}


\end{document}
